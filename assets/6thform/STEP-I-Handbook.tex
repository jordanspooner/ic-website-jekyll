\documentclass[9pt]{extarticle}
\title{The STEP I Handbook}
\date{\vspace{-12ex}}

\usepackage{geometry}
\geometry{a5paper, margin=0.35in, includefoot}

\usepackage{titlesec}
\newcommand{\sectionbreak}{\clearpage}
\renewcommand\thesubsection{\Alph{subsection}}
\setlength{\parskip}{1em}

\usepackage{graphicx}
\usepackage[justification=centering]{caption}
\usepackage{enumerate}
\usepackage{floatpag}
\usepackage{afterpage}
\usepackage{makebox}
\PassOptionsToPackage{hyphens}{url}\usepackage{hyperref}
\usepackage{amsmath}
\usepackage{amsfonts}
\usepackage{tabu}
\usepackage{booktabs}

\begin{document}
	
	\thispagestyle{empty}
	
	{\huge Notes for STEP I} \hfill \emph{(Last Updated May 2016)}
	
	General tips for preparation:
	\begin{enumerate}[(a)]
		\item Know the content. Make sure you are completely familiar with C3 and C4 before you even start to attempt STEP I questions. Likewise, make sure you know S2 and M2 before attempting applied questions. You need to be \emph{really} good at A-Level maths. If you can't get full marks (or close to it) in all of your A-Level modules, focus on that first.
		\item Don't shy away from difficult questions. Many students simply go through papers year by year, attempting only the six easiest questions (in their eyes). This is not a great method. Revise by topic instead. The most difficult questions are often the most useful. Go through all of the recent questions in each topic, until you feel sufficiently prepared to tackle any question in that topic. This way, you may also find that some questions are easier than they look.
		\item Don't look up the answers until you've finished the question (and checked it carefully). If you get stuck, leave the question and come back to it later. You will learn better (and get a greater sense of achievement) if you work it out yourself. An important skill for STEP is problem solving. Copying mark schemes will not help you to develop this skill.
	\end{enumerate}
	
	During the exam:
	\begin{enumerate}[(a)]
		\item Choose the questions for which you are quickly able to see a method for producing the answer. Look for problems that ask you to show or prove something. \textbf{Read the question} and check that you've actually answered it.
		\item Work slowly and methodically. \textbf{Check carefully} (and often). Remember, you have three hours to produce four good solutions. There's no need to rush.
		\item Don't panic. If you get stuck halfway through a question, don't be afraid to leave it and come back to it later with a clear head (also remember to take a look at \textbf{previous parts} of the question, which might give you some hints on how to progress).
	\end{enumerate}
	
	This booklet should be accompanied by `STEP I Questions since 2005'. The order of topics and questions is the same here as it is there, and the general hints and tips are written with the relevant questions in mind.
	
	Over the page, there is a table of questions included in the question booklet. Use this to check your solutions \emph{and} the examiner's report for that question. These will help you to gain a greater understanding of what the examiners were looking for, and how difficult students found that question. Compare your methods and reflect. Where available, mark your response using the official mark scheme.
	
	If you run out of questions, don't be afraid to look back at papers from before 2005, whose questions have not been included here simply for the purpose of brevity.
	
	\section*{Resources for STEP Revision}
	
	\thispagestyle{empty}
	
	\begin{enumerate}[(i)]
		\item STEP I Questions since 2005 by topic and this booklet.
		
		\url{http://jordanspooner.com/resources/}
		
		\item Past papers since 1987 and solutions since 1996.
		
		\url{http://www.mathshelper.co.uk/oxb.htm}
		
		\item Past papers since 1998; hints and solutions since 2004; examiners' reports since 2007; mark schemes (currently available for 2011 and 2013 - 2015 only).
		
		\url{http://www.admissionstestingservice.org/for-test-takers/step/preparing-for-step/}
		
		\item Advanced Problems in Mathematics.*
		
		\url{http://www.openbookpublishers.com/download/book/444}
		
		\item STEP correspondence course assignments.*
		
		\url{https://correspondence.maths.org/assignments}
		
		\item NRICH STEP preparation course (notes and some worked examples).
		
		\url{http://nrich.maths.org/11472}
		
		\item TSR example solutions.
		
		\url{http://www.thestudentroom.co.uk/wiki/Step_solutions}
		
		\item Meikleriggs example solutions.
		
		\url{http://meikleriggs.org.uk/}
		
		\item University of Warwick example solutions.
		
		\url{http://www2.warwick.ac.uk/fac/sci/maths/admissions/ug/aeastep/}
		
		\item Siklos' example solutions.
		
		\url{http://www.damtp.cam.ac.uk/user/stcs/STEP.html}
	\end{enumerate}
	
	\noindent Items with a * give you plenty of guidance and are highly recommended (possibly before you start attempting full STEP questions).
	
	\paragraph{A quick note about the following pages} The advice that follows will probably seem unnecessary to many readers. They are simply points I have identified that are common pitfalls when working through the questions myself. There is also some advice of particular use for the partnering questions. Feel free to remove the points you find trivial and add any points that you feel I've missed (the \LaTeX\ source is available on my website).
	
	Also please kindly send an email to \texttt{jordan@jordanspooner.com} if you notice any glaring errors (I'm sure they exist!) or if you think I've missed something of particular importance.
	
	\section*{Table of Questions}
	
	\thispagestyle{empty}
	
	\begin{tabu} to \linewidth {X[0.2]X[2]X[2]X[2]X[4]}
		\toprule
		& Topic & STEPC & APM & STEP I 2005-2015 \\
		\midrule
		1 & Integration & 5, 19, 24, 25, 28 & 9, 26, 32, 38, 39, 40, 41, 43, 49, 51 & 2005 1, 2006 1 and 6, 2007 1, 2008 1, 2009 1, 2010 8, 2011 8, 2014 1, 2015 8 \\
		2 & Number Theory and Combinatorics & 6, 10, 12, 18, 29 & 1, 2, 4, 11, 19, 30, 37 & 2005 1, 2006 1 and 6, 2007 1, 2008 1, 2009 1, 2010 8, 2011 8, 2014 1, 2015 8 \\
		3 & Calculus & 1, 3, 9, 13, 14, 17, 21, 23, 27, 32, 33 & WP1, 20, 36 & 2006 4, 2008 4, 2009 2 and 5, 2011 1 and 5, 2012 1, 2014 4, 2015 7 \\
		4 & Coordinate Geometry & 3, 4, 8, 20, 23, 26 & 23 & 2005 2 and 6, 2008 7, 2009 8, 2011 4, 2012 4, 2013 5, 2014 8, 2015 4 \\
		5 & General Probability & 12, 28, 30, 31, 32, 33 & 65, 66, 70, 71, 73 & 2005 12, 2006 12 and 14, 2007 12 and 13, 2008 13, 2011 12, 2013 12 \\
		6 & Projectiles & 26, 30 & 58 & 2006 10, 2007 11, 2008 10, 2009 9, 2011 9, 2012 9, 2014 9, 2015 9 \\
		7 & Collisions & 29 & 60 & 2005 10, 2006 11, 2009 11, 2010 11, 2011 10, 2013 10, 2014 10 \\
		8 & Graph Sketching & 7, 13, 19, 22 & 14, 15, 24, 33, 50 & 2006 3, 2007 8, 2009 3, 2010 2, 2012 2, 2013 2, 2015 1 \\
		9 & Discrete Random Variables & 27, 29 & 67, 68, 69 & 2007 13, 2010 12 and 13, 2012 13, 2013 13, 2014 12, 2015 12 \\
		10 & Vectors & 31, 33 & 47 & 2005 11, 2007 7, 2010 7 and 10, 2013 3, 2014 7, 2015 6 \\
		11 & Differential Equations & 27, 33 & 21, 44, 52 & 2005 8, 2008 8, 2010 6, 2011 7, 2012 8, 2013 7 \\
		12 & Algebra & 2, 7, 11, 15 & 5, 7, 10, 13, 16, 35, 45, 48 & 2005 3, 2007 4 and 6, 2010 1, 2013 1 and 9 \\
		13 & Trigonometry & 10, 17, 30 & 8, 34 & 2005 3, 2007 2, 2010 3, 2011 3, 2012 6, 2015 2 \\
		14 & Geometry & 1, 9, 31 & 6, 12, 42 & 2006 2 and 8, 2007 5, 2009 4, 2015 3 \\
		15 & Kinematics & 28, 32 & 53, 54, 55, 62 & 2006 9, 2007 10, 2012 10, 2013 11, 2014 11 \\
		16 & Statics & 27 & 56, 57, 59, 61, 63, 64 & 2005 9, 2008 11, 2010 9, 2011 11, 2015 11 \\
		17 & Continuous Random Variables & 26 & 72, 74, 75 & 2005 13 and 14, 2011 13, 2012 12, 2014 13 \\
		18 & Inequalities & 8 & 17, 18, 25, 27 & 2008 3, 2009 12, 2012 3, 2014 5 \\
		19 & Series and Sequences & 15, 16, 28 & WP2, 3, 22, 28, 29, 31, 46 & 2005 7, 2011 6, 2012 7, 2013 6 \\
		\bottomrule
	\end{tabu}

	\section*{Setting out Work and Making Fewer `Silly' Mistakes}
	
	\thispagestyle{empty}
	
	\paragraph{Write out More, not Less}
	
	\begin{itemize}
		\item You will make less mistakes by writing out factorisations, solutions to equations, the chain rule, product rule, quotient rule, integration by parts etc. over several lines rather than doing it in your head.
		\item You will be less likely to try a poor method if you write out what you're going to do before you do it.
		\item You are less likely to miscopy an equation / miscalculate a value if you write in a clear and structured manner and box or star important equations for later reference.
	\end{itemize}
	
	\paragraph{Checking your Work}
	
	\begin{itemize}
		\item It's probably a good idea to read back each line after you write it. Otherwise you could end up wasting a lot of time or losing a lot of marks for making simple errors.
		\item You should definitely check every discrete `section' of working (every time you deduce a significant result).
	\end{itemize}
	
	\paragraph{Reading the Question}
	
	\begin{itemize}
		\item Read the question until you fully understand it. Then attempt the question.
		\item Read the question again if you get stuck.
		\item Read the question again at the end and make sure you've given what was asked for.
	\end{itemize}
	
	\begin{figure}[p]
		\thisfloatpagestyle{empty}
		\begin{center}
			\includegraphics[scale=0.5]{strategy}
		\end{center}
	\end{figure}
	
	\clearpage
	\pagenumbering{arabic}
	
	\section{Integration}
	
	\begin{itemize}
		\item Make good use of \textbf{substitutions} (for integrals of the form $\int f(g(x))g'(x) \,dx$). If asked to identify a substitution, remember to consider its derivative and importantly \textbf{check the limits}. Often \textbf{hints} will be given earlier in the question. Also remember to check you've applied the correct \textbf{derivative of the substitution}.
		\item Sometimes the `correct' substitution may be obtained more naturally by \textbf{multiple substitutions} applied consecutively.
		\item Consider \textbf{partial fractions} (if the denominator can be factorised) or \textbf{dividing through} if you have a quotient.
		\item Consider \textbf{adding and subtracting} a value to make your integral nicer. E.g. $(x)(x+a)^n = (x+a-a)(x+a)^n = (x+a)^{n+1} - a(x+a)^n$. The same method is particularly useful for fractions.
		\item Use the \textbf{formula booklet} wisely. Occasionally you may spot a differential that allows the integral to be determined `by recognition'.
		\item Take care with \textbf{integration by parts}. It can be easy to slip up: check using the formula booklet. Always check for a substitution or manipulation before leaping into parts.
		\item Consider using \textbf{complex identities} (otherwise \textbf{trig identities}) for trigonometric functions, such as the $(z \pm 1/z)^n$ method for powers or moving between trig and exponential forms for a product of trig and exponential functions.
		\item \textbf{Experiment} with different methods at different points in the question (E.g. before and after applying trig identities).
		\item Consider \textbf{working backwards} (i.e. to prove an indefinite integral gives a desired result, differentiate that result).
		\item Consider the \textbf{graph} of your function. This can help to check results. Particularly useful when comparing multiple definite integrals from one question.
		\item Take care when evaluating definite integrals. \textbf{Check signs} and try not to ignore `trivial' terms without considering them carefully.
		\item Use \textbf{log rules} to simplify your answer but remember to watch out for $\ln 0$.
		\item Remember the \textbf{absolute value signs} for $\int f'(x)/f(x) \,dx = \ln|f(x)|$ and $\sqrt{f(x)^2} = |f(x)|$.
		\item Justify everything - especially when it's to do with signs (e.g. dropping \textbf{absolute value signs} or considering \textbf{square roots}).
	\end{itemize}
	
	\section{Number Theory and Combinatorics}
	
	\begin{itemize}
		\item To find all possible factors for an integer, consider its \textbf{prime factorization}. Don't forget the \textbf{four trivial factors}, which are $\pm N$ and $\pm 1$ for the integer $N$.
		\item Know that given a prime factorisation, $p^a \times q^b \times r^c \times \dots$, the \textbf{number of factors} is $(a+1)(b+1)(c+1)\dots$.
		\item Remember that for $a \mid n$ and $b \mid n$ implies $ab \mid n$ only if $a$ and $b$ are \textbf{co-prime}.
		\item If you are given that $a$ and $b$ are co-prime, considering $a/b$ may be helpful.
		\item Consider using \textbf{modular arithmetic}. E.g. to prove $5 \mid f(x) \forall x \in \mathbb{Z}$, consider the equation $f(x) = 0 \mod{5}$. 
		\item Know that \textbf{Fermat's Little Theorem} states that $a^p = a \mod{p}$ or equivalently $a^{p-1} = 1 \mod{p}$ for a prime $p$, as long as $a \not= 0 \mod{p}$.
		\item Consider similarities on each side of a diophantine equation: e.g. \textbf{factors, parity, perfect squares / cubes} to help you find possible solutions.
		\item To find patterns, start by testing \textbf{small cases}. Once you have an idea, try both to prove and to disprove your hypothesis (by considering \textbf{counterexamples}).
		\item Spend time to work out the \textbf{edge cases} and any \textbf{special cases}. Make sure you've included them all.
		\item Split long proofs into \textbf{individual cases}. E.g. to prove at most one of W, X, Y or Z is true, you can consider the six possible pairs and prove that no single pair of arguments can simultaneously be true (this kind of proof requires \textbf{careful planning}).
		\item Use \textbf{proof by contradiction} instinctively. If working with true statements doesn't produce what you're looking for, try working with a false statement instead.
		\item Try to generally \textbf{avoid proof by induction}, as there is often an easier method available.
		\item To prove X iff Y, you need to prove \textbf{both directions}: $X \implies Y$ and $Y \implies X$.
		\item Use any \textbf{hints} available to you from earlier in the question.
		\item \textbf{Work systematically} and \textbf{check} your answers very carefully.
	\end{itemize}

	\section{Calculus}
	
	\begin{itemize}
		\item \textbf{Draw diagrams} and \textbf{sketch graphs} to help understand the question.
		\item Know your graphs for trig functions. Know the \textbf{gradients} of $\sin x$, $\cos x$ and $\tan x$ at key points (so you can reason that, say, $\sin x = x$ has a unique solution at $x=0$).
		\item When presented with lots of different trig functions, try \textbf{factoring out} into $\cos x$ and $\sin x$ and see if it can be rearranged in a nicer form (also remember to watch out for opportunities to apply \textbf{trig identities}).
		\item Always \textbf{keep constants separate} when differentiating.
		\item Know that the \textbf{product rule} can be extended to $(uvw...)' = u'vw...+ uv'w... + uvw'...$.
		\item To find a second differential, consider expressing $f'(x)$ as a product of $f(x)$ and using the product rule.
		\item It is often worth considering \textbf{implicit differentiation} where there are, say, powers of both $x$ and $y$, or to avoid square roots.
		\item Remember there are multiple methods for proving a minimum or maximum. Consider the options carefully and choose the \textbf{simplest method} (evaluating second differential, or considering a point on each side).
		\item When given an interval, remember a function can have a min / max within that interval or on either boundary. You need to carefully \textbf{test every case}.
		\item If the question 1has \textbf{specific limits}, consider why. Never drop the given limits without reason.
		\item When solving $f(x)=0$, \textbf{always factorize} instead of dividing through (otherwise you lose potential solutions).
		\item When \textbf{factorising}, look for statements such as $x \not= y$ which suggests $(x-y)$ can be cancelled (and always justify why you can do this).
		\item Know the formulae for \textbf{volumes of revolution}: $V = \pi \int_{x_2}^{x_1} y^2 dx$ and $V = \pi \int_{y_2}^{y_1} x^2 dy$.
	\end{itemize}
	
	\section{Coordinate Geometry}
	
	\begin{itemize}
		\item Draw a \textbf{large diagram}.
		\item \textbf{Read the question} carefully and use all the given information. It is especially important to \textbf{consider conditions} such as $0<a<1$.
		\item Use the \textbf{formula booklet}. The equation for the perpendicular distance between a line and a point is often particularly useful.
		\item Look out for \textbf{symmetries}, both in your diagrams, and in your equations (e.g. can you replace all $x$'s with $y$'s for the same result?).
		\item Don't be afraid to use \textbf{vector notation} - it can often make the algebra easier.
		\item Remember to \textbf{compare coefficients} when two curves are found to be identical.
		\item Always try to \textbf{factorise} as early as possible. There are often marks for fully factorising important results.
		\item Know that the \textbf{difference of two squares / cubes} gives $(x^2-y^2) = (x-y)(x+y)$ and $(x^3 - y^3) = (x-y)(x^2 + xy + y^2)$ (the important factor here being $x-y$).
		\item You might need to \textbf{introduce a point} (such as $(a,b)$) in order to get an equation for a tangent or a normal (do not use $x$ and $y$!).
		\item Find ways to \textbf{convert between variables}. Then use these to your advantage (for example by removing a variable completely).
		\item To prove two \textbf{lines are perpendicular}, prove the product of their gradients is -1.
	\end{itemize}
	
	\section{General Probability}
	
	\begin{itemize}
		\item Draw a \textbf{tree diagram} or a \textbf{Venn diagram}.
		\item Look out for \textbf{conditional probabilities}, and know the following results (which are both in the formula booklet):
		
		\begin{align*}
		P(A \cup B) = P(A) + P(B) - P(A \cap B) && \text{(addition rule)} \\
		P(A \cap B) = P(A) \times P(B \mid A) && \text{(multiplication rule)}
		\end{align*}
		
		Calculate the probabilities individually and take care to ensure they meet the definition exactly.
		\item Where possible, \textbf{form an equation} of probabilities (by addition and multiplication rules or sum of probabilities = 1) and solve algebraically.
		\item It is often easiest to \textbf{make a combinatorial argument} to find probabilities.
		\item Know the standard methods for a `\textbf{slot filling}' approach to combinatorics and that for an arrangement of $k$ elements from $n$ elements:
		
		\begin{align*}
			&&							&&	\text{Order matters (list)} 	&& 	\text{Order doesn't matter (set)} 		\\
			&&	\text{Repetition} 		&& 	n^k 							&& 	\\
			&&	\text{No repetition} 	&& 	P(n,k) = \frac{n!}{(n-k)!} 		&& 	C(n,k) = \frac{n!}{(n-k)! k!} 			\\
		\end{align*}
		
		\item It is sometimes easier to calculate the \textbf{total number of arrangements and subtract} rather than calculating directly (or to \textbf{add up} different arrangements).
		\item Try to \textbf{minimise the number of arrangements} you work with. I.e. use combinations instead of permutations where possible, exclude rotations, etc. But remember to keep it consistent.
		\item It is important to work methodically \textbf{check carefully} that you haven't under-counted or over-counted (Did I miss something? Does the same arrangement appear twice?).
		\item It is a good idea to check your answers by \textbf{testing small cases} (for general probabilities) or \textbf{checking every case} and adding all probabilities to ensure they sum to 1.
		\item \textbf{Explain} carefully what you are doing (be prepared to write a lot!).
	\end{itemize}
	
	\section{Projectiles}
	
	\begin{itemize}
		\item Always \textbf{draw a diagram}.
		\item It is worth learning that $\boxed{y = x\tan\alpha - \frac{gx^2}{2V^2\cos^2\alpha}}$.
		\item You may also wish to learn the following results (all of which are easily derived from VUSAT equations):
		\begin{align*}
		t = \frac{2V\sin\alpha}{g} && \text{(time of flight)} \\
		H = \frac{V^2\sin^2\alpha}{2g} && \text{(maximum height)} \\
		R = \frac{V^2\sin 2\alpha}{g} && \text{(horizontal range)}
		\end{align*}
		\item State when you are using a \textbf{trig identity} and make sure it is used correctly.
		\item Make sure you \textbf{know the meanings} (including directions!) of any variables you use (occasionally you are given a variable in the question and you will have to work out its relevance to the question).
		\item When using the substitution method for solving a simultaneous equation, consider carefully the \textbf{easiest substitution} (e.g. avoid solving for $t$ if there is acceleration).
		\item Be very careful in justifying \textbf{positive or negative roots}.
	\end{itemize}
	
	\section{Collisions}
	
	\begin{itemize}
		\item Always \textbf{draw a diagram}.
		\item Define clearly the \textbf{positive direction} and take care with signs.
		\item Take care with \textbf{handwriting}, especially for $v$ and $V$, $u$ and $U$, etc.
		\item Be prepared to consider \textbf{overall changes} in momentum and kinetic energy.
		\item It is often easier to \textbf{work with energy} than to use VUSAT equations for constant acceleration.
	\end{itemize}
	
	\section{Graph Sketching}
	
	\begin{itemize}
		\item Consider \textbf{intercepts}, \textbf{asymptotes}, \textbf{symmetries}, and \textbf{stationary points}. Clearly show your working for all four; there are often marks allocated, say, for differentiating the function to find stationary points, even if this is not explicitly mentioned.
		\item State \textbf{exact $x$ and $y$ co-ordinates} for any important features on your graph. There are often marks for these even if they are not explicitly asked for.
		\item Look out for \textbf{transformations} such as $f(-x)$, $f(x^2)$ or $\pm\sqrt{f(x)}$, and think carefully about what effect these transformations have on the $x$ or $y$ values.
		\item When using the factor theorem, it is worth considering where there might be a \textbf{sign change}. Don't forget to consider negative or simple fractional factors.
		\item \textbf{Completing the square} can be an efficient way to find minimums / maximums.
		\item \textbf{Check} factorisations, solutions, etc. very carefully. It is also sometimes worth checking your graph by substituting in some easy $x$ values.
	\end{itemize}
	
	\section{Discrete Random Variables}
	
	\begin{itemize}
		\item Use the \textbf{formula booklet}!
		\item Know when (and when not!) to use \textbf{binomial}, \textbf{poisson} and \textbf{geometric} distributions.
		\item \textbf{Think carefully about probabilities}. Think about whether they are conditional, and whether there is an easier way to find them (e.g. $1-p$). Make sure you haven't over or under-counted.
		\item For probability distributions and expected values, consider every possible outcome and its probability (\textbf{draw a table}!). Don't group outcomes together.
	\end{itemize}
	
	\section{Vectors}
	
	\begin{itemize}
		\item Read the question carefully and draw a \textbf{correct diagram}.
		\item To prove $\mathbf{a}$ and $\mathbf{b}$ are \textbf{perpendicular} simply show that $\mathbf{a}.\mathbf{b}=0$ (remember you can also use the \textbf{dot product} to find lengths and angles: $|\mathbf{a}||\mathbf{b}|\cos\theta = \mathbf{a}.\mathbf{b}$).
		\item To prove $\begin{pmatrix}x_1\\y_1\end{pmatrix}$ and $\begin{pmatrix}x_2\\y_2\end{pmatrix}$ are \textbf{parallel}, it is easier to show that $\frac{x_1}{y_1}=\frac{x_2}{y_2}$.
		\item The position vector, $1/5 \mathbf{a} + 4/5 \mathbf{b}$ divides $AB$ in the \textbf{ratio} 4:1.
		\item It is useful to know that the \textbf{distance between two lines} is given by $\frac{|(a_1-a_2).(d_1 \times d_2)|}{|d_1 \times d_2|}$.
	\end{itemize}
	
	\section{Differential Equations}
	
	\begin{itemize}
		\item \textbf{Rate} means you should be considering time.
		\item Don't forget to consider your \textbf{integration constant}!
		\item Don't forget to comment on \textbf{domains} where relevant (particularly square roots).
	\end{itemize}
	
	\section{Algebra}
	
	\begin{itemize}
		\item \textbf{Factorise by recognition}. Know the difference of two squares / cubes.
		\item Know how to expand an expression using \textbf{Pascal's triangle}. Know also how to use \textbf{repeated squaring} to expand an expression efficiently.
		\item Write out \textbf{all working}. For example, when comparing coefficients, show all of your equations and resulting deductions and state explicitly that your results are consistent. Don't make big leaps; you need to explain every step carefully.
		\item Check your working and, importantly, \textbf{check your solutions}.
	\end{itemize}
	
	\section{Trigonometry}
	
	\begin{itemize}
		\item Watch out for \textbf{signs}!
		\item Consider \textbf{dividing through by an identity} deduced earlier in the question.
		\item Make sure you can identify when to use \textbf{compound angle formulae} and their results, such as:
		$$\cos2\theta = 1-2\sin^2\theta = 2\cos^2\theta-1$$
		$$\cos^2\theta = 1/2 + 1/2\cos2\theta$$
		$$\sin^2\theta = 1/2 - 1/2\cos2\theta$$
	\end{itemize}
	
	\section{Geometry}
	
	\begin{itemize}
		\item It is important to \textbf{read the question} particularly carefully and \textbf{draw a clear diagram}.
		\item If asked to \textbf{prove} a result, make sure you do so! Consider carefully the easiest method.
		\item Be prepared to \textbf{use results from earlier} in the question in your arguments.
	\end{itemize}
	
	\section{Kinematics}
	
	\begin{itemize}
		\item Take time to think carefully about the motion and forces and draw a \textbf{large clear diagram}.
		\item When considering the motion of a system, it is often a good idea to \textbf{consider the motion of each object separately}.
		\item Remember to consider the \textbf{overall acceleration}. For example, for a system within a system, you will have to consider the sum of two accelerations (one for the object within its system and one for that system in the overall system).
	\end{itemize}
	
	\section{Statics}
	
	\begin{itemize}
		\item Take time to think carefully about the forces and draw a \textbf{large clear diagram}.
		\item \textbf{Don't assume $F = \mu R$}. If it is not given that the object is about to slip, then you should instead state $F \leq \mu R$.
		\item If you're unsure what to do, try \textbf{resolving twice and taking moments}.
		\item Know that \textbf{three non-parallel forces} on a large body in equilibrium are \textbf{concurrent}.
		\item It is worth knowing \textbf{Lami's theorem}, which states that $\frac{A}{\sin\alpha} = \frac{B}{\sin\beta} = \frac{C}{\sin\gamma}$ where $A$, $B$ and $C$ are forces on an object in equilibrium and $\alpha$, $\beta$ and $\gamma$ are the angles opposite the corresponding forces.
	\end{itemize}
	
	\section{Continuous Random Variables}
	
	\begin{itemize}
		\item Know that the \textbf{probability density function} is the derivative of the \textbf{continuous distribution function}.
		\item Be prepared to draw \textbf{tree diagrams}.
	\end{itemize}
	
	\section{Inequalities}
	
	\begin{itemize}
		\item It is often easier to \textbf{prove $x - y > 0$ instead of $x > y$}.
		\item Need to be careful with \textbf{possible negatives} and \textbf{squares}.
		\item Know the \textbf{triangle inequality} states that $|x|+|y| \geq |x+y| \geq |x|-|y|$ (this should be fairly intuitive).
		\item Know the \textbf{AM-GM inequality} states that $\frac{x_1+x_2+...+x_n}{n} \geq \sqrt{x_1\times x_2\times ...\times x_n}$.
		\item Know the \textbf{Cauchy-Schwarz inequality} states that $|x_1y_1 + x_2y_2 + x_3y_3 + \dots + x_ny_n| \leq \sqrt{x_1^2 + x_2^2 + \dots + x_n^2} \sqrt{y_1^2 + y_2^2 + \dots + y_n^2}$.
	\end{itemize}
	
	\section{Series and Sequences}
	
	\begin{itemize}
		\item When proving a result about any numbers from two sequences, \textbf{don't work with corresponding terms} (i.e. use $n$th term from A and $k$th term from B).
		\item It is occasionally worth considering \textbf{proof by induction}.
	\end{itemize}
	
\end{document}